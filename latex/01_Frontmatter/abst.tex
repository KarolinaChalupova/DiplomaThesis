\section*{Abstract}

This thesis trains and interprets 4 neural networks used for stock return prediction. The models have state-of-the-art performance in terms of profitability. The 30 predictors studied are a distillation of 50 years of asset pricing literature. Out of them, financial constraints, behavioral biases of investors, risk of illiquidity and value effect appear to be the most important. The thesis is likely the first to show the \textit{direction} of the effects of these variables on stock returns in neural networks: the networks overwhelmingly agree in sign with the economic motivations of the predictors. The results are obtained using a metric that has ready interpretation and is well grounded in ML interpretability theory. In addition, a novel metric is proposed that offers insight into what drives the ability of networks to distinguish the best- from worst-performing stocks in the market. In particular, it shows which variables the long-short portfolios constructed using the networks rely on the most. Comparing the interpretations to linear regression points to which variables likely have an important non-linear relationship to returns. Further decomposing the interpretation in time and into constituents of ensemble models draws an even more detailed picture of variable importance. The thesis unveils the drivers behind the outstanding performance of networks in stock returns prediction and shows that machines are able to give much more useful answers than "42".  

\bigskip

\begin{tabular}{lp{8.6cm}}
		\textbf{JEL Classification} & \JEL \\
		\textbf{Keywords} & \Keywords \\
 		& \\
		\textbf{Title} & \Bookname \\
 		\textbf{Author's e-mail} & \texttt{\href{mailto:\Email}{\Email}}\\
		\textbf{Supervisor's e-mail} & \texttt{\href{mailto:\EmailSup}{\EmailSup}}\\
		\textbf{Consultant's e-mail} & \texttt{\href{mailto:\EmailCon}{\EmailCon}}\\
\end{tabular}

\bigskip

\section*{Abstrakt}\label{abstract}

TODO cesky preklad abstraktu. 

\bigskip

\begin{tabular}{lp{7.7cm}}
		\textbf{Klasifikace JEL} & \JEL \\
		\textbf{Kl\'{i}\v{c}ov\'{a} slova} & \Klic \\
 		& \\
		\textbf{N\'{a}zev pr\'{a}ce} & \BooknameCZ \\
 		\textbf{E-mail autora} & \texttt{\href{mailto:\Email}{\Email}}\\
		\textbf{E-mail vedouc\'{i}ho pr\'{a}ce} & \texttt{\href{mailto:\EmailSup}{\EmailSup}}\\
		\textbf{E-mail konzultanta pr\'{a}ce} & \texttt{\href{mailto:\EmailCon}{\EmailCon}}\\
\end{tabular}

