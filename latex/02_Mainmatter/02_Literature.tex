\chapter{Literature Review}
\label{chap:lit} 
	
 	
 	The field of asset pricing strives to put a value, or a price tag, on uncertain future payoffs. 
 	
 	One of the most studied problems in finance is explaining why different assets earn different returns. The problem is so central that an entire field evolved to solve it – the asset pricing. 
 	
 	
 	Answering this question is vital, as by doing so, one also explains many crucial phenomena. Why do some stock prices rise, while other stocks become worthless? What drives the returns - is it exposure to risk, market imperfections, mispricing or random chance? Is it all of them? Are some of the drivers more important than others? Is a given stock priced fairly, or is it too expensive or too cheap? How does the stock market go up and down? Why does it go mostly up? Why do some stocks go up in bad times? Are returns predictable? To what degree? Why do stock market bubbles occur? The answer to the over-arching problem of why different assets earn different returns would also explain all of these questions.  
 	
 	The question is so important that the in
 	
 	Asset pricing theory gives the following theoretical answer. Return of stock 
 	
 	\begin{equation}
 	r_t = \beta_{t-1}F_t + \epsilon_t  = g(C_{t-1})F_t + \epsilon_t \label{eq:asset_pricing}
 	\end{equation}
 	
 	The canonical answer to this question is that expected return of an asset at time $t$ is a linear function of systematic sources of risk, so-called factors ($F_t$). $\beta_{t-1}$, also called factor loadings, sensitivity to factors, or just betas, denote the amount of exposure to the underlying sources of risk: 
		
		\begin{equation}
			r_t = \beta_{t-1}F_t + \epsilon_t  = g(C_{t-1})F_t + \epsilon_t \label{eq:asset_pricing}
		\end{equation}
		
	
	
	
	$\beta_{t-1}$ is a $N \times K$ matrix, where $N$ denotes the number of assets and $K$ the amount of risk factors. $F_t$ is a $K \times1 $ vector.
	
	An important implication of \ref{eq:asset_pricing} is that the factor loadings, in turn, are a possibly non-linear function of firm-level characteristics ($C_{t-1}$). 
		
		\begin{equation}
			\beta_{t-1}= g(C_{t-1})\label{eq:characteristics_as_proxies}
		\end{equation}
	
	In the words of \cite{fama1993common}: "(...) if assets are priced rationally, variables that are
	related to average returns, such as size and book-to-market equity, must proxy for
	sensitivity to common (shared and thus undiversiable) risk factors in returns." 	
	
	The theoretical answer is therefore clear: the differences in average returns are explained by different exposure to risk factors. 
	Take the simplest example, the CAPM model, where $K=1$ and $F_t$ is a (scalar) return of market portfolio at time $t$ \citep{cochrane2009asset}: on average, a high return on a stock is just a compensation for the stock's high correlation $\beta_{t-1}$ with the market. 
	
	The empirical answer, however, is complicated. One empirical issue with estimating equation \ref{eq:asset_pricing} is that both $F_t$ and $\beta_{t-1}$ are unobservable \citep{kelly2019characteristics}. The question reduces to: how to obtain the factors? 
	
	The first approach is to use prior knowledge of empirical behavior of average returns to pre-specify factors $F_t$, treat them as known and observable and then estimate $\beta_{t-1}$. In the portfolio-sorting method, a characteristic is chosen (somehow). Stocks are then sorted into portfolios based on their value of this characteristic \citep{fama1993common}. It is then studied whether different returns of these portfolios can be explained by a simpler factor model, such as CAPM. Using this mehod, academia has generated about 300 factors \citep{harvey2016and}, considering the top journals only.
	
	\cite{bryzgalova2019forest} generalize the portfolio-sorts using random trees.     
	
	The second strand of literature treats both $F_t$ and $\beta_{t-1}$ as unobservable, and estimates them both from the data, using the  relationship that firm-level characteristics are proxies of factor loadings (\ref{eq:characteristics_as_proxies}). For example, \cite{kelly2019characteristics} use characteristics as instrumental variables for factor loadings. Once loadings are instrumented, they use them to estimate the corresponding factors. 
	
    
	
	

	
	   
	
	   
	



