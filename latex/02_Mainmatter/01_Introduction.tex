\chapter{Introduction}
\label{chap:int}

Understanding of the drivers of stock prices is important for a number of reasons. First, fair stock valuation is vital for proper functioning of the stock market. The stock market, in turn, needs to function so as to maintain its roles: it enables firms to obtain financing for their investments, it allows investors to store their present wealth for the future and to share risk. [TODO elaborate] Moreover, as history has shown, incorrect stock price valuations can have severe ramifications. [TODO explain this more and give examples.]

Over the last 50 years, the academia has accumulated hundreds of variables that are proposed to explain stock returns. \cite{harvey2016and} and \cite{mclean2016does} have shown that most of these existing research findings are likely false and the field entered a deep crisis. The multidimensional challenge \citep{cochrane2011presidential} emerged: which of the published and unpublished determinants of stock returns are valid, and which are erroneous? The search for a reliable model is, essentially, same as in any other field. First, the explanatory variables should be based on theory. Second, rigorous statistical methods should be used to ensure robustness of the model. Specifically, if there are very many potential explanatory variables, it is necessary to consider all of them jointly, that is, to control for the rest of the variables. It is possible to include all the candidate variables in a single model, allowing the effects to crowd each other out. However, with hundreds of candidate variables the asset pricing field has accumulated over decades, traditional methods break down. $R^2$ goes very deep to the negative territory in unpenalized linear regressions \citep{gu2020empirical} and portfolio-sorts become unusable as early as with 4 variables. This is where ML methods come in. 

Machine learning is on the rise in the asset pricing field, both in academia and industry. The academia is using machine learning (ML) methods to tackle the biggest challenges of the field, such as the problem of high dimensionality. Meanwhile, in the industry, finance practitioners use unparalleled predictive power of ML methods to forecast asset returns \citep{gu2020empirical}. [TODO add evidence of ML rise].    

However, there is a trade-off between model performance and its intrinsic interpretability, which is why ML models are often perceived as black-box. [TODO develop further]

The solution to overcome this trade-off is to use ML interpretability methods to extract insights about the model ex-post. This can be readily done even for more complex models such as neural networks. [TODO develop further]

Interpretable ML can be used as the driving force of scientific discovery in many fields, including asset pricing. It can be used to propose plausible causes of phenomena, i.e., to help find the possible $X$ in $y=f(X)$ problems. Typically, human researchers propose the $X$, and then devise a cause-effect mechanism and test it. The proposal of $X$ can be done using interpretable ML. This is already done in practice [TODO examples from chemistry or others]. [TODO explain usefullness to asset pricing academia].
 
Interpretability of a ML model is also crucial for the asset-pricing industry. First, the typical approach to model discovery, that is, trying different models and backtesting them on the same data, leads to false positive discovery: it only takes 20 trials to find a false positive at the standard 5\% significance, in other words, it is quite easy to overfit the backtest. Understading the sources of model performance can be used as an alternative research tool to guide strategy discovery \citep{de2018advances}: for example, using feature importance measures, one uncover which inputs to the model are important, and use this knowledge to limit the amount of noise or add more strong predictors. Second, knowing the sources of model performance can help with marketability of the model, as when ML interpretability methods are used, the model ceases to be a black box. From a practitioner’s perspective, my thesis can serve as a case study of some of the methods, issues, and results that occur when interpreting return-predicting ML models.

The objective of this thesis is \ldots. [TODO]

The thesis is structured as follows: \ldots [TODO] 
